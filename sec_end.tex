
\section{Replication team at Cornell University}

\subsection{Replicators} 
\label{app:replicators}

The following \textbf{\teamsize} students have provided excellent assistance in reproducing the results from the \jiramcs{} articles processed by the Replication Lab:
%
% Pulled from processing-jira...
%
%
Adam J. Faridi,
Akshay Yadava,
Alice Wei,
Amie Li,
Ananya Bakshi,
Andrew Phiri,
Andrew Wallace,
Anjini Khanna,
Anurag Tiwari,
Arnaav Sareen,
Bianca Jimenez,
Caitlin Song,
Crystal Lim,
Elian Gomez,
Ethan Carlson,
Gary Wu,
Hawi Tolera,
Ilona Khimey,
Jade Yang,
Jaeyoung Shim,
Jason Lan,
Jessica Rizzo,
Joshua Wallace,
Kareena Stowers,
Kate Chanpong,
Kayla Yang,
Kirin Eicher,
Kristine Li,
Leslie Geng,
Lincy Chen,
Luke Trautwein,
Manvir Chahal,
Melanie Brown,
Micere Mugweru,
Miranda Zhou,
Nguyen Vo,
Olivia Liu,
Phalguni Miraj,
Sherry Li,
Siddhi Malvankar,
Sohit Gurung,
Talia Boehm,
Tommy Wang,
Vidya Balaji,
Yuchang Tian.

%
Graduate students  Leonel Borja Plaza and Linda Wang and Research Aide Sofia Encarnación (all Cornell University) have been invaluable assistants in training and coordinating the work as well as developing the methods and procedures which we have made public.  Linda Wang contributed programming to this report. Sofia Encarnación contributed to all parts of the process. A description and evaluation of the training of replicators for the AEA's Data Editor team was published as \citet{vilhuber_teaching_2022}. The training and reference manual can be found at \purlcite{https://labordynamicsinstitute.github.io/ldilab-manual/}{online}

\subsection{Computing support}

We thank the Economics Department and the ILR School for providing us with computing resources at the Cornell Center for Social Sciences and the Bioinformatics cluster. 

\section{Third-party contributors}

\subsection{Replicators}
\label{app:3rdparty}

We are grateful to the  third-party replicators who assisted us with verifications when we were unable to access data or, in some cases, computing resources. 
%We do not name individuals when doing so would reveal information not already known to the manuscript's authors, naming the institution instead. Names are listed in no particular order.
%
Graduate research assistants at Stanford, Penn State University, and at Cornell helped us. We thank  Paulo Guimarães and staff at \ac{BPLIM} who contributed their  time to  run code  on confidential data and provide us with detailed knowledge about the data being used. We in particular want to again thank Olivier Akmansoy, Christophe Hurlin (Université d'Orléans), and Christophe Pérignon (HEC Paris), all of \href{https://cascad.tech}{cascad}, a certification agency for scientific code and data, who have been generous of their time and resources, and have provided us with multiple reports during this time.  

We do not name the authors with whom we signed non-disclosure agreements, or who otherwise provided us with access to data that could not be published. We are grateful for their flexibility and patience.

\subsection{Computing resources}
\label{app:3rdparty-computing}

We are grateful to Codeocean, NBER, WholeTale, and Harvard Business School, who all provided us with access to computing resources at no cost, and technical assistance when necessary. We use free academic resources on Github and Bitbucket. WholeTale is free to use for any academic user.


\section{Disclosures}
\label{sec:disclosure}

We received a generous compute and storage quota from \href{https://codeocean.com/}{Codeocean}, a free license to use Stata 17/18 for one year in cloud applications from \href{https://stata.com/}{Stata}, and a subaward on NSF grant \href{https://nsf.gov/awardsearch/showAward?AWD_ID=1541450&HistoricalAwards=false}{1541450} ``CC*DNI DIBBS: Merging Science and Cyberinfrastructure Pathways: The Whole Tale'' from the University of Illinois to evaluate the WholeTale platform for the purpose of reproducibility verification. None of the sponsors have reviewed this preliminary assessment, or have had influence on any of the conclusions of this document. Codeocean currently offers academic users a certain number of monthly free compute hours. WholeTale is free to use.


\section{Data and Code Availability Statement}
\label{sec:dcas}

All publicly available data and code used to generate figures and tables in this article are available \citep{report2023data,E117876V5}. Some detailed data from the editorial system, used for Table~\ref{tab:pre:rounds}, are considered confidential and cannot be made available in a way that preserves the privacy of the editorial process at this time.
